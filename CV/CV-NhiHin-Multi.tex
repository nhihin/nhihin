\documentclass[10pt,a4paper,ragged2e,withhyper]{altacv2}
\geometry{left=1.4cm,right=1.4cm,top=2cm,bottom=1.5cm,columnsep=1cm}
\usepackage{paracol}
\usepackage[default]{roboto}

% Change the colours if you want to
\definecolor{VividPurple}{HTML}{000000}
\definecolor{SlateGrey}{HTML}{2E2E2E}
\definecolor{LightGrey}{HTML}{555555}
\definecolor{PaleGrey}{HTML}{777777}
% \colorlet{name}{black}
\colorlet{tagline}{black}
\colorlet{heading}{black}
\colorlet{headingrule}{black}
% \colorlet{subheading}{PastelRed}
\colorlet{accent}{black}
\colorlet{emphasis}{SlateGrey}
\colorlet{body}{LightGrey}
\colorlet{palegrey}{PaleGrey}

% Change the bullets for itemize and rating marker
% for \cvskill if you want to
\renewcommand{\itemmarker}{{\small\textbullet}}
\renewcommand{\ratingmarker}{\faCircle}


\begin{document}
\name{Nhi Hin}
\personalinfo{%
  \mailaddress{hello@nhihin.com}
  \phone{+61 424 875 797}
  \linkedin{nhihin}
  \twitter{nhihin}
  \github{nhihin}
}
\makecvheader
\medskip\medskip

\tolerance=9999
\emergencystretch=10pt
\hyphenpenalty=10000
\exhyphenpenalty=100

%% Depending on your tastes, you may want to make fonts of itemize environments slightly smaller
\AtBeginEnvironment{itemize}{\small}

%% Set the left/right column width ratio
\columnratio{0.7}

% Start a 2-column paracol. Both the left and right columns will automatically
% break across pages if things get too long.
\begin{paracol}{2}

% ---------------------------------------------------------------- %
%                            EXPERIENCE                            %
% ---------------------------------------------------------------- %
\cvsection{Experience}

\cvevent{Bioinformatician}{South Australian Genomics Centre (SAGC)}{August 2020 -- Ongoing}{Adelaide, Australia}
\begin{itemize}
\item Developed custom workflows and pipelines to enable reproducible processing and analysis of single-cell, spatial, and bulk transcriptomic data.
\item Integrated single-cell datasets from multiple tissues (skin and heart) by applying appropriate statistical and machine learning techniques in R and Python.
\item Applied trajectory and velocity analysis techniques to identify and characterise novel progenitor cell types during tissue development and wounding. 
\item Deployed web applications for visualising large-scale single-cell datasets using R/Shiny and Google Cloud Platform.
\item Consulted with remote clients and collaborators on diverse multi-omics problems, and quoted for, implemented and delivered 8+ tailored bioinformatics solutions within time and budgetary constraints.
\end{itemize}

% ---------------------------------------------------------------- %
%                            EDUCATION                             %
% ---------------------------------------------------------------- %

\cvsection{Education}
\cvevent{Doctor of Philosophy (Bioinformatics)}{The University of Adelaide}{2017 -- 2020}{}
\begin{itemize}
  \item \yearbold{Thesis Title:} Transcriptome analysis of zebrafish genetic models to reveal early molecular drivers of Alzheimer's disease
\end{itemize}

\medskip
\cveventalt{Key Projects}{Iron Responsive Element (IRE)-Mediated Responses to Iron Dyshomeostasis in Alzheimer's Disease}{Jun 2019 -- Aug 2020}{R, shell scripting, Snakemake, High Performance Computing, R/Bioconductor, limma and edgeR, GSEA, workflowR}
\begin{itemize}
\item Developed workflow for exploring iron homeostasis at the transcriptional level and applied this to analyze zebrafish, mouse, and human datasets.
% \item Adapted cutting-edge single-cell velocity analysis technique to explore transcript stability in bulk RNA-seq dataset.
\item Presented findings at International Joint GIW-ABACBS Conference (Sydney, Dec 2019), winning Best Student Presentation Award (1st Prize). 
\item \boldline{Hin N}, Newman M, Pederson S, Lardelli M. Iron Responsive Element Mediated Responses to Iron Dyshomeostasis in Alzheimer’s Disease. \textit{J. Alzheimer's Dis.} 2021;84(4):1597-630.
\end{itemize}

\cveventalt{}{Accelerated brain aging towards transcriptional inversion in a zebrafish model of the K115fs mutation of human \textit{PSEN2}}{Jun 2017 -- May 2019}{R, shell scripting, High Performance Computing, WGCNA, Cytoscape, R/Bioconductor, limma}
\begin{itemize}
\item Performed co-expression network analysis to compare brain transcriptomes in a familial Alzheimer's disease animal model with sporadic Alzheimer's disease in humans.
\item Applied multivariate statistics and modelling to analyse bulk RNA-seq and microarray datasets. 
\item Designed and created visualisation-rich presentations that won 4 awards from multiple nationally-recognised conferences and symposiums.
\item \boldline{Hin N*}, Newman M*, Kaslin J, Douek AM, Lumsden A, Nik SHM, Dong Y, Zhou XF, Mañucat-Tan NB, Ludington, A, Adelson DL, Pederson S, Lardelli M. Accelerated brain aging towards transcriptional inversion in a zebrafish model of the K115fs mutation of human \textit{PSEN2}. \textit{PLoS One} 2020;15(1):p.e0227258. 
\end{itemize}

\cvsection{Education (cont.)}
\cvevent{Bachelor of Science (Advanced)}{The University of Adelaide}{2014 -- 2016 | \yearbold{GPA:} 6.8 / 7}{}
\begin{itemize}
  \item \yearbold{Majors:} Genetics and Chemistry
\end{itemize}

% ---------------------------------------------------------------- %
%                            PUBLICATIONS                          %
% ---------------------------------------------------------------- %

\cvsection{Publications}

* indicates shared first-authorship
\medskip

\begin{itemize}
  \item \boldline{Hin N}, Newman M, Pederson S, Lardelli M. Iron Responsive Element Mediated Responses to Iron Dyshomeostasis in Alzheimer’s Disease. \textit{J. Alzheimer's Dis.} 2021;84(4):1597-630.
  \item \boldline{Hin N*}, Newman M*, Kaslin J, Douek AM, Lumsden A, Nik SHM, Dong Y, Zhou XF, Mañucat-Tan NB, Ludington, A, Adelson DL, Pederson S, Lardelli M. Accelerated brain aging towards transcriptional inversion in a zebrafish model of the K115fs mutation of human \textit{PSEN2}. \textit{PLoS One} 2020;15(1):p.e0227258. 
  \item Newman M*, \boldline{Hin N*}, Pederson S, Lardelli M. Brain transcriptome analysis of a familial Alzheimer’s disease-like mutation in the zebrafish presenilin 1 gene implies effects on energy production. \textit{Mol. Brain} 2019;12(1):1-5. 
  \item Dong Y, Newman M, Pederson S, Barthelson K, \boldline{Hin, N}, Lardelli, M. Transcriptome analyses of 7-day-old zebrafish larvae possessing a familial Alzheimer’s disease-like mutation in \textit{psen1} indicate effects on oxidative phosphorylation, ECM and MCM functions, and iron homeostasis. \textit{BMC Genom.} 2021;22(1):1-16. 
  \item Breen J, McAninch D, Jankovic-Karasoulos T, McCullough D, Smith MD, Bogias KJ, Wan Q, Choudhry, A, \boldline{Hin N}, Pederson SM, Bianco-Miotto T. Temporal placental genome wide expression profiles reflect three phases of utero-placental blood flow during early to mid human gestation. \textit{medRxiv} 2020 (Pre-Print).
  \item Newman M, Nik HM, Sutherland GT, \boldline{Hin N}, Kim WS, Halliday, GM, Jayadev S, Smith C, Laird AS, Lucas CW,  Kittipassorn T. Accelerated loss of hypoxia response in zebrafish with familial Alzheimer’s disease-like mutation of presenilin 1. \textit{Hum. Mol. Genet.} 2020;29(14):2379-94.
\end{itemize}

\switchcolumn

% ---------------------------------------------------------------- %
%                            OBJECTIVE                             %
% ---------------------------------------------------------------- %

\cvsection{Objective}
Australian Bioinformatician with 4+ years of experience in transcriptomics and solid background in Alzheimer's disease research. Seeking to bring expertise in single-cell and spatial transcriptomics to Alkahest Inc. Highly skilled in statistics, data visualization, and creative thinking. 

% ---------------------------------------------------------------- %
%                            ACHIEVEMENTS                          %
% ---------------------------------------------------------------- %
\cvsection{Awards}
\begin{itemize}
  \item \customsmalltext{\yearbold{2021 - Recipient of 10X-Millenium Science Spatial Pioneers Fellowship Scheme}. Australia.}
  \smallskip
  \item \yearbold{2021 - Semi-Finalist in Channel 7 7NEWS Young Achiever Awards}. Adelaide, Australia.
  \smallskip
  \item \yearbold{2020 - Doctoral Research Medal (top 4\% of theses submitted in 2020)}. The University of Adelaide. Adelaide, Australia.
  \smallskip
  \item \yearbold{2020 - Best Student Presentation Award (1st Place), out of 15 speakers}, Joint GIW-ABACBS International Conference. Sydney, Australia.
  \smallskip
  \item \yearbold{2019 - Best Poster Presentation Award, out of 35 entrants}, AMSI BioInfoSummer. Sydney, Australia.
  \smallskip
  \item \yearbold{2019 - Best Presentation Award, out of 50 speakers}, Australia-Japan Joint Neurodegenerative Disease Symposium. Adelaide, Australia.
  \smallskip
  \item \yearbold{2019 - CHOOSEMaths Travel Grant Winner, one of 20 winners selected nationally}, Australian Mathematical Sciences Institute (AMSI). Sydney, Australia.
  \smallskip
  \item \yearbold{2017 - Oral Presentation Award (3rd Place), out of 15 speakers}, COMBINE Symposium. Adelaide, Australia.
  \smallskip
  \item \yearbold{2017 - Best Conference Poster Award, out of 36 entrants}, Model Organisms in Human Health Australia (MOHHA). Yarra Valley, Victoria.
\end{itemize}

%\cvsection{Peer Review Experience}
%\begin{itemize}
%\item \yearbold{2021} -- Invited reviewer for \textit{PLOS One}
%\item \yearbold{2021} -- Invited reviewer for \textit{International Journal of General Medicine}
%\end{itemize}
%\medskip


\newpage

% ---------------------------------------------------------------- %
%                            TEACHING                              %
% ---------------------------------------------------------------- %

\cvsection{Teaching Experience}
\cvevent{}{Tutor for South Australian Genomics Centre Workshop Series}{August 2021}{}
\begin{itemize}
\item Developed and taught workshop material for class of 100+ attendees. 
\item \yearbold{Topics:} Bulk RNA-seq Analysis, Single-cell RNA-seq Analysis
\end{itemize}
\divider
\cvevent{}{Tutor for Spring Into Bioinformatics Workshop}{September 2019}{}
\begin{itemize}
\item Developed and taught workshop material for class of 50+ students, as part of the University of Adelaide's Bioinformatics Hub Workshop Series.
\item \yearbold{Topics:} Introduction to R and RStudio for Biological Research; Gene Expression Analysis for RNA-seq Data; QC and Alignment for Next-Generation Sequencing Data
\end{itemize}




\switchcolumn

% ---------------------------------------------------------------- %
%                       CONFERENCE PRESENTATIONS                   %
% ---------------------------------------------------------------- %

\cvsection{Conference Presentations}
\textbf{\smallsubsection{Talks}}
\begin{itemize}
    \item RNA-seq analysis of a zebrafish model of Alzheimer’s disease reveals the importance of iron homeostasis. Presented at \textit{COMBINE Symposium} and \textit{Joint GIW-ABACBS International Conference}. December 2019. Sydney, Australia. (\textbf{Oral Presentation Prize / 1st place}, out of 15 speakers)
    \item Computational Analysis Reveals an Early Brain Iron Deficiency Response in a Zebrafish Mutation Model of Familial Alzheimer’s Disease. Presented at the \textit{Back to Basics: Understanding the Molecular Basis of Alzheimer’s Disease workshop}. September 2019. Sydney, Australia.
    \item Bioinformatics analysis of familial Alzheimer’s disease-like zebrafish. Presented at the \textit{AustraliaJapan Neurodegenerative Disease Symposium}. June 2019. Adelaide, Australia. (\textbf{Best Presentation Award}, out of 50 speakers)
    \item Diving into Alzheimer’s Disease with Transcriptome Analysis of a Zebrafish Model. Presented at \textit{COMBINE Symposium}. October 2017. Adelaide, Australia. (\textbf{Oral Presentation Prize / 3rd place}, out of 15 speakers)
\end{itemize}

\divider

\textbf{\smallsubsection{Posters}}
\begin{itemize}
    \item Iron Enriched. Presented at \textit{AMSI BioInfoSummer}, December 2019. Sydney, Australia. (\textbf{Best Poster Prize}, out of 35 entrants)
    \item RNA-seq analysis of aging and Alzheimer’s disease in a zebrafish model. Presented at the \textit{Model Organisms for Human Health Australia (MOHHA) Conference}, June 2017. Melbourne, Victoria. (\textbf{Best Poster Prize}, out of 36 entrants)
\end{itemize}
\medskip

\end{paracol}
\newpage

% ---------------------------------------------------------------- %
%                            SKILLS                                %
% ---------------------------------------------------------------- %

\cvsection{Scientific and Technical Expertise}

\columnratio{0.5}
\begin{paracol}{2}

\yearbold{\smallsubsection{Programming Languages}}
\begin{itemize}
    \item R, including use of tidyverse, Bioconductor, shiny, and workflowR
    \item Python, including use of NumPy, pandas, and scikit-learn
    \item Shell scripting
\end{itemize}
\medskip

\yearbold{\smallsubsection{Bioinformatics Data Processing and Analysis}}
\begin{itemize}
    \item Single-cell RNA-seq
    \item Spatial transcriptomics
    \item Bulk RNA-seq
    \item Microarrays
    \item CRISPR screens
    \item LS-MS/MS proteomics
    \item Metagenomics
    \item Whole genome bisulfite sequencing
    \item Whole genome sequencing
    \item Analysis of gene co-expression networks
    \item Accessing and integrating data from biological databases
    \item Multi-omics analysis and data integration
    \item Experimental design and power calculations
\end{itemize}
\medskip

\yearbold{\smallsubsection{Software Engineering and Data Management}}
\begin{itemize}
    \item Use of high performance computing (HPC) resources via SLURM
    \item Development of pipelines/workflows using Snakemake, Nextflow, and workflowR
    \item Use of Git and version control
    \item Use of Docker containers and Conda environments
    \item Development of unit tests
\end{itemize}
\medskip

\switchcolumn

\yearbold{\smallsubsection{Machine Learning and Data Science}}
\begin{itemize}
    \item Data cleaning and pre-processing
    \item Dealing with missing or censored values and technical batch effects
    \item Feature selection and engineering
    \item Supervised analysis problems, including classification and regression
    \item Unsupervised analysis problems, including clustering, matrix factorisation, and dimension reduction
    \item Network analysis and graph visualisation
    \item Web app development using R/Shiny
    \item Deployment of applications on Google Cloud Platform
    \item Use of SQL databases
\end{itemize}
\medskip

\yearbold{\smallsubsection{Other Technical Skills}}
\begin{itemize}
    \item Adobe Photoshop and InDesign for scientific illustration and poster design
    \item HTML/CSS and website management
\end{itemize}
\medskip
\end{paracol}

\columnratio{0.5}

% ---------------------------------------------------------------- %
%                            REFERENCES                            %
% ---------------------------------------------------------------- %

\cvsection{Referees}
\begin{paracol}{2}

    \cvref{Dr.\ Jimmy Breen}{Former Head of Bioinformatics,\\  South Australian Genomics Centre}{jimmy.breen@adelaide.edu.au}{+61 477 830 366}
    {}
\switchcolumn
    \cvref{A/Prof.\ Michael Lardelli}{The University of Adelaide}{michael.lardelli@adelaide.edu.au}{(08) 8313 3212}
    {}

\end{paracol}


\end{document}
